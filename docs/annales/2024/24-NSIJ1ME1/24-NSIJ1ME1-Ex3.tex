

\emph{Cet exercice porte sur la programmation orientée objet, sur les
arbres binaires de recherche et la récursivité.}

Chaque année, plusieurs courses de chiens de traîneaux sont organisées
sur les terrains enneigés. L'une d'elle, \emph{La Traversée Blanche},
est une course se déroulant en 9 étapes. L'organisateur de cette course
est chargé de créer un programme Python pour aider à la bonne gestion de
l'événement.

\subsubsection{Partie A : la classe
Chien}\label{partie-a-la-classe-chien}

Afin de caractériser un chien, l'organisateur décide de créer une classe
\texttt{Chien} avec les attributs suivants : - \texttt{id\_chien}, un
nombre entier correspondant au numéro attribué au chien lors de son
inscription à la course ; - \texttt{nom}, une chaîne de caractères
correspondant au nom du chien ; - \texttt{role}, une chaîne de
caractères correspondant au poste occupé par le chien : en fonction de
sa place dans l'attelage, un chien a un rôle bien défini et peut être
\texttt{\textquotesingle{}leader\textquotesingle{}},
\texttt{\textquotesingle{}swing\ dog\textquotesingle{}},
\texttt{\textquotesingle{}wheel\ dog\textquotesingle{}} ou
\texttt{\textquotesingle{}team\ dog\textquotesingle{}}. -
\texttt{id\_proprietaire}, un nombre entier correspondant au numéro de
l'équipe.

Le code Python incomplet de la classe \texttt{Chien} est donné
ci-dessous.

\begin{Shaded}
\begin{Highlighting}[]
\DecValTok{1} \KeywordTok{class}\NormalTok{ Chien:}
\DecValTok{2}   \KeywordTok{def} \FunctionTok{\_\_init\_\_}\NormalTok{(}\VariableTok{self}\NormalTok{, id\_chien, nom, role, id\_prop):}
\DecValTok{3}       \VariableTok{self}\NormalTok{.id\_chien }\OperatorTok{=}\NormalTok{ id\_chien}
\DecValTok{4}       \VariableTok{self}\NormalTok{.nom }\OperatorTok{=}\NormalTok{ nom}
\DecValTok{5}       \VariableTok{self}\NormalTok{.role }\OperatorTok{=}\NormalTok{ role}
\DecValTok{6}       \VariableTok{self}\NormalTok{.id\_proprietaire }\OperatorTok{=}\NormalTok{ id\_prop}
\DecValTok{7}   \KeywordTok{def}\NormalTok{ changer\_role(}\VariableTok{self}\NormalTok{, nouveau\_role):}
\DecValTok{8}       \StringTok{"""Change le rôle du chien avec la valeur passée en}
\StringTok{            paramètre."""}
\DecValTok{9}\NormalTok{ ...}
\end{Highlighting}
\end{Shaded}

Voici un extrait des informations dont on dispose sur les chiens
inscrits à la course.

\includegraphics{24-NSIJ1ME1-Ex3-01.png}

Suite aux inscriptions, l'organisateur procède à la création de tous les
objets de type \texttt{Chien} et les stocke dans des variables en
choisissant un nom explicite. Ainsi, l'objet dont l'attribut
\texttt{id\_chien} a pour valeur 40 est stocké dans la variable
\texttt{chien40}.

\begin{enumerate}
\def\labelenumi{\arabic{enumi}.}
\tightlist
\item
  \textbf{Écrire} l'instruction permettant d'instancier l'objet
  \texttt{chien40} caractérisant le chien ayant le numéro d'inscription
  40.
\item
  Selon l'état de fatigue de ses chiens ou du profil de l'étape, le
  \emph{musher} (nom donné à la personne qui conduit le traîneau) peut
  décider de changer le rôle des chiens dans l'attelage.
\end{enumerate}

\textbf{Recopier} et \textbf{compléter} la méthode
\texttt{changer\_role} de la classe \texttt{Chien}.

\begin{enumerate}
\def\labelenumi{\arabic{enumi}.}
\setcounter{enumi}{2}
\tightlist
\item
  Le propriétaire de Duke décide de lui attribuer le rôle de
  \texttt{\textquotesingle{}leader\textquotesingle{}}.
\end{enumerate}

\textbf{Écrire} l'instruction permettant d'effectuer cette modification.

\subsubsection{Partie B : la classe
Equipe}\label{partie-b-la-classe-equipe}

On souhaite à présent créer une classe \texttt{Equipe} ayant les
attributs suivants :

\begin{itemize}
\tightlist
\item
  \texttt{num\_dossard}, un nombre entier correspondant au numéro
  inscrit sur le dossard du musher ;
\item
  \texttt{nom\_equipe}, une chaîne de caractères correspondant au nom de
  l'équipe ;
\item
  \texttt{liste\_chiens}, une liste d'objets de type \texttt{Chien} dont
  chaque élément correspond à un chien au départ de l'étape du jour ;
\item
  \texttt{temps\_etape}, une chaîne de caractères (par exemple
  \texttt{\textquotesingle{}2h34\textquotesingle{}}) représentant le
  temps mis par l'équipe pour parcourir l'étape du jour ;
\item
  \texttt{liste\_temps}, une liste de chaînes de caractères permettant
  de stocker les temps de l'équipe pour chacune des 9 étapes. Cet
  attribut peut, par exemple, contenir la liste :
  \texttt{{[}\textquotesingle{}4h36\textquotesingle{},\ \textquotesingle{}3h57\textquotesingle{},\ \textquotesingle{}3h09\textquotesingle{},\ \textquotesingle{}5h49\textquotesingle{},\ \textquotesingle{}4h45\textquotesingle{},\ \textquotesingle{}3h26\textquotesingle{},\ \textquotesingle{}4h57\textquotesingle{},\ \textquotesingle{}5h52\textquotesingle{},\ \textquotesingle{}4h31\textquotesingle{}{]}}.
\end{itemize}

On donne le code Python suivant de la classe \texttt{Equipe}.

\begin{Shaded}
\begin{Highlighting}[]
\DecValTok{1} \KeywordTok{class}\NormalTok{ Equipe:}
\DecValTok{2}   \KeywordTok{def} \FunctionTok{\_\_init\_\_}\NormalTok{(}\VariableTok{self}\NormalTok{, num\_dossard, nom\_equipe):}
\DecValTok{3}       \VariableTok{self}\NormalTok{.num\_dossard }\OperatorTok{=}\NormalTok{ num\_dossard}
\DecValTok{4}       \VariableTok{self}\NormalTok{.nom\_equipe }\OperatorTok{=}\NormalTok{ nom\_equipe}
\DecValTok{5}       \VariableTok{self}\NormalTok{.liste\_chiens }\OperatorTok{=}\NormalTok{ []}
\DecValTok{6}       \VariableTok{self}\NormalTok{.temps\_etape }\OperatorTok{=} \StringTok{\textquotesingle{}\textquotesingle{}}
\DecValTok{7}       \VariableTok{self}\NormalTok{.liste\_temps }\OperatorTok{=}\NormalTok{ []}
\DecValTok{8}
\DecValTok{9}   \KeywordTok{def}\NormalTok{ ajouter\_chien(}\VariableTok{self}\NormalTok{, chien):}
\DecValTok{10}      \VariableTok{self}\NormalTok{.liste\_chiens.append(chien)}
\DecValTok{11}
\DecValTok{12}  \KeywordTok{def}\NormalTok{ retirer\_chien(}\VariableTok{self}\NormalTok{, numero):}
\DecValTok{13}\NormalTok{      ...}
\DecValTok{14}
\DecValTok{15}  \KeywordTok{def}\NormalTok{ ajouter\_temps\_etape(}\VariableTok{self}\NormalTok{, temps):}
\DecValTok{16}      \VariableTok{self}\NormalTok{.liste\_temps.append(temps)}
\end{Highlighting}
\end{Shaded}

Pour la première étape, le musher de l'équipe numéro 11, représentée en
Python par l'objet \texttt{eq11}, décide de constituer une équipe avec
les quatre chiens identifiés par les numéros 42, 44, 45 et 46. On donne
ci-dessous les instructions Python permettant de créer l'équipe
\texttt{eq11} et l'attelage constitué des 4 chiens précédents.

\begin{Shaded}
\begin{Highlighting}[]
\DecValTok{1}\NormalTok{ eq11 }\OperatorTok{=}\NormalTok{ Equipe(}\DecValTok{11}\NormalTok{, }\StringTok{\textquotesingle{}Malamutes Endurants\textquotesingle{}}\NormalTok{)}
\DecValTok{2}\NormalTok{ eq11.ajouter\_chien(chien42)}
\DecValTok{3}\NormalTok{ eq11.ajouter\_chien(chien44)}
\DecValTok{4}\NormalTok{ eq11.ajouter\_chien(chien45)}
\DecValTok{5}\NormalTok{ eq11.ajouter\_chien(chien46)}
\end{Highlighting}
\end{Shaded}

Malheureusement, le musher s'aperçoit que sa chienne Helka, chien numéro
46, n'est pas au mieux de sa forme et il décide de la retirer de
l'attelage. 4. \textbf{Recopier} et \textbf{compléter} la méthode
\texttt{retirer\_chien} ayant pour paramètre numero, un entier
correspondant au numéro attribué au chien lors de l'inscription, et
permettant de mettre à jour l'attribut \texttt{liste\_chiens} après
retrait du chien dont la valeur de l'attribut \texttt{id\_chien} est
numero. 5. En vous aidant de la fonction précédente, écrire
l'instruction qui permet de retirer Helka de l'attelage de l'équipe
\texttt{eq11}.

On donne à présent le code Python d'une fonction \texttt{convert}
prenant pour paramètre \texttt{chaine}, une chaîne de caractères
représentant une durée, donnée en heure et minute.

On supposera que cette durée est toujours strictement inférieure à 10
heures, temps maximal fixé par le règlement pour terminer une étape.

\begin{Shaded}
\begin{Highlighting}[]
\DecValTok{1} \KeywordTok{def}\NormalTok{ convert(chaine):}
\DecValTok{2}\NormalTok{       heure\_dec }\OperatorTok{=} \BuiltInTok{int}\NormalTok{(chaine[}\DecValTok{0}\NormalTok{]) }\OperatorTok{+} \BuiltInTok{int}\NormalTok{(chaine[}\DecValTok{2}\NormalTok{] }\OperatorTok{+}\NormalTok{ chaine[}\DecValTok{3}\NormalTok{])}\OperatorTok{/}\DecValTok{60}
\DecValTok{3}       \ControlFlowTok{return}\NormalTok{ heure\_dec}
\end{Highlighting}
\end{Shaded}

\begin{enumerate}
\def\labelenumi{\arabic{enumi}.}
\setcounter{enumi}{5}
\tightlist
\item
  \textbf{Indiquer} le résultat renvoyé par l'appel
  \texttt{convert(\textquotesingle{}4h36\textquotesingle{})}.
\item
  \textbf{Écrire} une fonction \texttt{temps\_course} qui prend pour
  paramètre \texttt{equipe} de type \texttt{Equipe} et qui renvoie un
  nombre flottant correspondant au cumul des temps de l'équipe
  \texttt{equipe} à l'issue des 9 étapes de la course.
\end{enumerate}

On rappelle que la classe \texttt{Equipe} dispose d'un attribut
\texttt{liste\_temps}.

\subsection{Partie C : classement à l'issue d'une
étape}\label{partie-c-classement-uxe0-lissue-dune-uxe9tape}

Chaque jour, à la fin de l'étape, on décide de construire un Arbre
Binaire de Recherche (ABR) afin d'établir le classement des équipes.
Chaque nœud de cet arbre est un objet de type \texttt{Equipe}.

Dans cet arbre binaire de recherche, en tout nœud :

\begin{itemize}
\tightlist
\item
  toutes les équipes du sous-arbre gauche sont strictement plus rapides
  que ce nœud ;
\item
  toutes les équipes du sous-arbre droit sont moins rapides ou sont à
  égalité avec ce nœud.
\end{itemize}

Voici les temps, en heure et minute, relevés à l'issue de la première
étape :

\includegraphics{24-NSIJ1ME1-Ex3-02.png}

Dans l'arbre binaire de recherche initialement vide, on ajoute
successivement, dans cet ordre, les équipes \texttt{eq1}, \texttt{eq2},
\texttt{eq3}, \ldots, \texttt{eq11}, 11 objets de la classe
\texttt{Equipe} tous construits sur le même modèle que l'objet
\texttt{eq11} précédent.

\begin{enumerate}
\def\labelenumi{\arabic{enumi}.}
\setcounter{enumi}{7}
\tightlist
\item
  Dans l'arbre binaire de recherche ci-dessous, les nœuds \texttt{eq1}
  et \texttt{eq2} ont été insérés.
\end{enumerate}

\textbf{Recopier} et \textbf{compléter} cet arbre en insérant les 9
nœuds manquants.

\includegraphics{24-NSIJ1ME1-Ex3-03.png}

\begin{enumerate}
\def\labelenumi{\arabic{enumi}.}
\setcounter{enumi}{8}
\item
  \textbf{Indiquer} quel parcours d'arbre permet d'obtenir la liste des
  équipes classées de la plus rapide à la plus lente.
\item
  On donne ci-dessous la classe \texttt{Noeud}, permettant de définir
  les arbres binaires :
\end{enumerate}

\begin{Shaded}
\begin{Highlighting}[]
\DecValTok{1} \KeywordTok{class}\NormalTok{ Noeud:}
\DecValTok{2}   \KeywordTok{def} \FunctionTok{\_\_init\_\_}\NormalTok{(}\VariableTok{self}\NormalTok{, equipe, gauche }\OperatorTok{=} \VariableTok{None}\NormalTok{, droit }\OperatorTok{=} \VariableTok{None}\NormalTok{):}
\DecValTok{3}       \VariableTok{self}\NormalTok{.racine }\OperatorTok{=}\NormalTok{ equipe}
\DecValTok{4}       \VariableTok{self}\NormalTok{.gauche }\OperatorTok{=}\NormalTok{ gauche}
\DecValTok{5}       \VariableTok{self}\NormalTok{.droit }\OperatorTok{=}\NormalTok{ droit}
\end{Highlighting}
\end{Shaded}

On donne ci-dessous le code d'une fonction \texttt{construction\_arbre}
qui, à partir d'une liste d'éléments de type \texttt{Noeud} permet
d'insérer successivement chaque nœud à sa place dans l'ABR.

\begin{Shaded}
\begin{Highlighting}[]
\DecValTok{1} \KeywordTok{def}\NormalTok{ construction\_arbre(liste):}
\DecValTok{2}\NormalTok{   a }\OperatorTok{=}\NormalTok{ Noeud(liste[}\DecValTok{0}\NormalTok{])}
\DecValTok{3}   \ControlFlowTok{for}\NormalTok{ i }\KeywordTok{in} \BuiltInTok{range}\NormalTok{(}\DecValTok{1}\NormalTok{,}\BuiltInTok{len}\NormalTok{(liste)):}
\DecValTok{4}\NormalTok{       inserer(a, liste[i])}
\DecValTok{5}   \ControlFlowTok{return}\NormalTok{ a}
\end{Highlighting}
\end{Shaded}

La fonction \texttt{construction\_arbre} fait appel à la fonction
\texttt{inserer} qui prend pour paramètre \texttt{arb}, de type
\texttt{Noeud}, et \texttt{eq}, de type \texttt{Equipe}. Cette fonction
construit le nœud à partir de \texttt{eq} et l'insère à sa place dans
l'ABR.

\begin{Shaded}
\begin{Highlighting}[]
\DecValTok{1} \KeywordTok{def}\NormalTok{ inserer(arb, eq):}
\DecValTok{2}   \StringTok{""" Insertion d\textquotesingle{}une équipe à sa place dans un ABR contenant}
\StringTok{3   au moins un noeud."""}
\DecValTok{4}   \ControlFlowTok{if}\NormalTok{ convert(eq.temps\_etape) }\OperatorTok{\textless{}}\NormalTok{ convert(arb.racine.temps\_etape):}
\DecValTok{5}       \ControlFlowTok{if}\NormalTok{ arb.gauche }\KeywordTok{is} \VariableTok{None}\NormalTok{:}
\DecValTok{6}\NormalTok{           arb.gauche }\OperatorTok{=}\NormalTok{ ...}
\DecValTok{7}       \ControlFlowTok{else}\NormalTok{:}
\DecValTok{8}\NormalTok{           inserer(..., eq)}
\DecValTok{9}   \ControlFlowTok{else}\NormalTok{:}
\DecValTok{10}      \ControlFlowTok{if}\NormalTok{ arb.droit }\KeywordTok{is} \VariableTok{None}\NormalTok{:}
\DecValTok{11}\NormalTok{          arb.droit }\OperatorTok{=}\NormalTok{ Noeud(eq)}
\DecValTok{12}      \ControlFlowTok{else}\NormalTok{:}
\DecValTok{13}\NormalTok{          ...}
\end{Highlighting}
\end{Shaded}

\textbf{Expliquer} en quoi la fonction \texttt{inserer} est une fonction
récursive. 11. \textbf{Recopier} et \textbf{compléter} les lignes 6, 8
et 13 de la fonction \texttt{inserer}. 12. \textbf{Recopier} et
\textbf{compléter} les lignes 3 et 5 de la fonction
\texttt{est\_gagnante} ci-dessous qui prend en paramètre un ABR
\texttt{arbre}, de type \texttt{Noeud}, et qui renvoie le nom de
l'équipe ayant gagné l'étape.

\begin{Shaded}
\begin{Highlighting}[]
\DecValTok{1} \KeywordTok{def}\NormalTok{ est\_gagnante(arbre):}
\DecValTok{2}     \ControlFlowTok{if}\NormalTok{ arbre.gauche }\OperatorTok{==} \VariableTok{None}\NormalTok{:}
\DecValTok{3}       \ControlFlowTok{return}\NormalTok{ ...}
\DecValTok{4}     \ControlFlowTok{else}\NormalTok{:}
\DecValTok{5}       \ControlFlowTok{return}\NormalTok{ ...}
\end{Highlighting}
\end{Shaded}

\subsubsection{Partie D : classement
général}\label{partie-d-classement-guxe9nuxe9ral}

On décide d'établir un classement général obtenu à partir du cumul des
temps mis par chaque équipe pour parcourir l'ensemble des 9 étapes.

Sur le même principe que l'arbre de la partie précédente, on construit
l'ABR ci-dessous qui permet, grâce au parcours d'arbre approprié,
d'établir ce classement général des équipes.

\includegraphics{24-NSIJ1ME1-Ex3-04.png}

Le règlement prévoit la disqualification d'une équipe en cas de
non-respect de celui-ci. Il s'avère que l'équipe 2 et l'équipe 5 doivent
être disqualifiées pour manquement au règlement. Les nœuds \texttt{eq2}
et \texttt{eq5} doivent donc être supprimés de l'ABR précédent.

Pour supprimer un nœud \texttt{N} dans un ABR, trois possibilités se
présentent : - le nœud \texttt{N} à supprimer est une feuille : il
suffit de le retirer de l'arbre ; - le nœud \texttt{N} à supprimer n'a
qu'un seul fils : on relie le fils de \texttt{N} au père de \texttt{N}
et on supprime le nœud \texttt{N} ; - le nœud \texttt{N} à supprimer
possède deux fils : on le remplace par son successeur (l'équipe qui a le
temps immédiatement supérieur) qui est toujours le minimum de ses
descendants droits.

\begin{enumerate}
\def\labelenumi{\arabic{enumi}.}
\setcounter{enumi}{12}
\tightlist
\item
  Dessiner le nouvel arbre de recherche \texttt{a\_final} obtenu après
  suppression des équipes \texttt{eq2} et \texttt{eq5} dans l'ABR
  correspondant au classement général.
\end{enumerate}

L'organisateur souhaite disposer d'une fonction rechercher permettant de
savoir si une équipe a été disqualifiée ou non. On donne les
spécifications de la fonction \texttt{rechercher}, prenant en paramètre
\texttt{arbre} et \texttt{equipe}.

\begin{Shaded}
\begin{Highlighting}[]
\DecValTok{1} \KeywordTok{def}\NormalTok{ rechercher(arbre, equipe):}
\DecValTok{2}   \StringTok{"""}
\StringTok{3       Paramètres}
\StringTok{4       {-}{-}{-}{-}{-}{-}{-}{-}{-}}
\StringTok{5           arbre : un ABR, non vide, de type Noeud, représentant le}
\StringTok{6           classement général.}
\StringTok{7           equipe : un élément, de type Equipe, dont on veut déterminer}
\StringTok{8           l\textquotesingle{}appartenance ou non à l\textquotesingle{}ABR arbre.}
\StringTok{9       Résultat}
\StringTok{10      {-}{-}{-}{-}{-}{-}{-}{-}{-}}
\StringTok{11          Cette fonction renvoie True si equipe est un nœud de arbre,}
\StringTok{12          False sinon.}
\StringTok{13 """}
\DecValTok{14}\NormalTok{ ...}
\end{Highlighting}
\end{Shaded}

Pour cette fonction (\texttt{a\_final} désigne l'arbre obtenu à la
question 13, après suppression des équipes 2 et 5) : - l'appel
\texttt{rechercher(a\_final,\ eq1)} renvoie \texttt{True} ; - l'appel
\texttt{rechercher(a\_final,\ eq2)} renvoie \texttt{False}.

\begin{enumerate}
\def\labelenumi{\arabic{enumi}.}
\setcounter{enumi}{13}
\tightlist
\item
  \textbf{Écrire} le code de la fonction
\end{enumerate}
